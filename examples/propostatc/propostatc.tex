\documentclass[propostatc,showlinks]{unisc}    % Opção showlinks exibe os links em cor azul

\usepackage[T1]{fontenc}                       % Suporte a acentuação no arquivo de saída
\usepackage[utf8]{inputenc}                    % Codificação dos arquivos de entrada em UTF-8
\usepackage[english,brazilian]{babel}          % Idiomas utilizados no trabalho (o último é o principal)
\usepackage{float}                             % Maior controle de objetos "float" (tabelas, figuras, etc.)

\title{Exemplo de Proposta de TCC da UNISC}    % Título do Trabalho
\author{Sobrenome}{Nome Segundo Nome}          % Autor do Trabalho
\advisor[Prof. Msc.]{Sobrenome}{Nome}          % Orientador
\reviewer[Prof. Msc.]{Sobrenome}{Nome}         % Avaliador 1
\reviewer[Prof. Msc.]{Sobrenome}{Nome}         % Avaliador 2
\reviewer[Prof. Msc.]{Sobrenome}{Nome}         % Avaliador 3

\dept{Departamento de Informática}
\course{Curso de Ciência da Computação}
\location{Santa Cruz do Sul}{RS}
\date{11}{setembro}{2015}


\begin{document}

\maketitle
\tableofcontents

\chapter*{Resumo}

Alguns orientadores preferem colocar Introdução, outros preferem Resumo.

\chapter*{Motivação}

Motivação para o trabalho

\chapter*{Objetivos}

O objetivo geral deste trabalho é demonstrar como criar uma proposta de trabalho de conclusão de curso utilizando o modelo \LaTeX da UNISC.

Os objetivos específicos são:

\begin{itemize}
  \item Promover o uso de \LaTeX na UNISC
  \item Outros objetivos específicos
\end{itemize}

\chapter*{Metodologia}

Características da pesquisa. Procedimentos metodológicos. Etapas do cronograma.

\begin{enumerate}
  \item Elaboração da proposta de TC
  \begin{enumerate}
    \item Definir o assunto
    \item Pesquisar trabalhos relacionados
    \item Escrita da proposta
  \end{enumerate}
  \item Outros itens...
\end{enumerate}

\chapter*{Cronograma}

A tabela do cronograma pode ser montada com \LaTeX e utilizar o modelo abaixo. Ou pode ser feita como uma imagem e apenas carregar o arquivo aqui.

\begin{table}[H]
  \label{tab:cronograma}
  \centering
  \begin{tabular}{c|c|c|c|c|c|c|c|c|c|c|c|c|c|}
    \cline{2-14} & \multicolumn{6}{c|}{2015} & \multicolumn{7}{c|}{2016} \\
    \cline{2-14} & Jul & Ago & Set & Out & Nov & Dez & Jan & Fev & Mar & Abr & Mai & Jun & Jul \\
    %                                      JUL AGO SET OUT NOV DEZ JAN FEV MAR ABR MAI JUN JUL
    \cline{1-14} \multicolumn{1}{|c|}{1}  & X & X &   &   &   &   &   &   &   &   &   &   &   \\
    \cline{1-14} \multicolumn{1}{|c|}{2}  &   & X &   &   &   &   &   &   &   &   &   &   &   \\
    \cline{1-14} \multicolumn{1}{|c|}{3}  &   & X & X &   &   &   &   &   &   &   &   &   &   \\
    \cline{1-14} \multicolumn{1}{|c|}{4}  &   &   & X & X &   &   &   &   &   &   &   &   &   \\
    \cline{1-14} \multicolumn{1}{|c|}{5}  &   &   & X & X &   &   &   &   &   &   &   &   &   \\
    \cline{1-14} \multicolumn{1}{|c|}{6}  &   &   & X & X & X &   &   &   &   &   &   &   &   \\
    \cline{1-14} \multicolumn{1}{|c|}{7}  &   &   &   & X & X &   &   &   &   &   &   &   &   \\
    \cline{1-14} \multicolumn{1}{|c|}{8}  &   &   &   & X & X & X &   &   &   &   &   &   &   \\
    \cline{1-14} \multicolumn{1}{|c|}{9}  &   &   &   &   & X & X &   &   &   &   &   &   &   \\
    \cline{1-14} \multicolumn{1}{|c|}{10} &   &   &   &   &   & X & X &   &   &   &   &   &   \\
    \cline{1-14} \multicolumn{1}{|c|}{11} &   &   &   &   &   &   & X & X &   &   &   &   &   \\
    \cline{1-14} \multicolumn{1}{|c|}{12} &   &   &   &   &   &   &   & X & X & X & X &   &   \\
    \cline{1-14} \multicolumn{1}{|c|}{13} &   &   &   &   &   &   &   &   &   & X & X & X &   \\
    \cline{1-14} \multicolumn{1}{|c|}{14} &   &   &   &   &   &   &   &   &   &   & X & X & X \\
    \cline{1-14} \multicolumn{1}{|c|}{15} &   &   &   &   &   &   &   &   &   &   &   &   & X \\
    \cline{1-14} \multicolumn{1}{|c|}{16} &   &   &   &   &   &   &   &   &   &   &   &   & X \\
    \cline{1-14}
  \end{tabular}
\end{table}

% Carrega as referências do arquivo referencias.bib
\bibliographystyle{abntex2-alf}
\bibliography{referencias}

% Opcional, dependendo do orientador
\makesignature

\end{document}
